%%%%%%%%%%%%%%%%%%%%%%%%%%%%%%%%%%%%%%%%%%%%%%%%%%%%%%%%%%%%%%%%%%%%%%%%%%%%%%%%%%%
%% This project aims to create the UFC template for presentation.                %%
%% author: Maurício Moreira Neto - Doctoral student in Computer Science (MDCC)   %%
%% contacts:                                                                     %%
%%    e-mail: maumneto@ufc.br                                                    %%
%%    linktree: https://linktr.ee/maumneto                                      %%
%%%%%%%%%%%%%%%%%%%%%%%%%%%%%%%%%%%%%%%%%%%%%%%%%%%%%%%%%%%%%%%%%%%%%%%%%%%%%%%%%%%
\documentclass{libs/ufc_format}

% Inserting the preamble file with the packages
%%%%%%%%%%%%%%%%%%%%%%%%%%%%%%%%%%%%%%%%%%%%%%%%%%%%%%%%%%%%%%%%%%%%%
%% This file contains the packages that can be used in the beamer. %%
%%%%%%%%%%%%%%%%%%%%%%%%%%%%%%%%%%%%%%%%%%%%%%%%%%%%%%%%%%%%%%%%%%%%%
% Package to accentuation
\usepackage[utf8]{inputenc}
% Package to Portuguese language
\usepackage[brazil]{babel}
% Package to Figures
\usepackage{graphicx}
% Package to the colors
\usepackage{color}
% Package to the colors
\usepackage{xcolor}
% Packages to math symbols and expressions
\usepackage{amsfonts, amssymb, amsmath}
% Package to multiple lines and columns in table
\usepackage{multirow, array} 
% Package to create pseudo-code
% For more detail of this package: http://linorg.usp.br/CTAN/macros/latex/contrib/algorithm2e/doc/algorithm2e.pdf
\usepackage{algorithm2e}
% Package to insert code
\usepackage{listings} 
\usepackage{keyval}
% Package with color for keywords in code
\usepackage{minted}


% creating a command for UFC
\newcommand{\ufc}{\normalsize{Universidade Federal do Ceará}}

% Title
\title[short title]{\textbf{Título da Apresentação}}
% Subtitle
\subtitle{Subtítulo da Apresentação}
% Author of the presentation
\author{Nome do Autor}
% Institute's Name
\institute[UFC]{
    \ufc
}
% date of the presentation
\date{\today}

% PDF's name 
\subject{document}

%%%%%%%%%%%%%%%%%%%%%%%%%%%%%%%%%%%%%%%%%%%%%%%%%%%%%%%%%%%%%%%%%%%%%%%%%%%%%%%%%%
%% Start Document of the Presentation                                           %%               
%%%%%%%%%%%%%%%%%%%%%%%%%%%%%%%%%%%%%%%%%%%%%%%%%%%%%%%%%%%%%%%%%%%%%%%%%%%%%%%%%%
\begin{document}
% insert the code style
%%%%%%%%%%%%%%%%%%%%%%%%%%%%%%%%%%%%%%%%%%%%%%%%%%%%%%%%%%%%%%%%%%%%%%%%%%%%%%%%%%%
%% This file contains the style of the codes show in slides.                     %%
%% The package used is listings, but it possible to used others.                 %%
%%%%%%%%%%%%%%%%%%%%%%%%%%%%%%%%%%%%%%%%%%%%%%%%%%%%%%%%%%%%%%%%%%%%%%%%%%%%%%%%%%%

% color used in the code style
\definecolor{codegreen}{rgb}{0,0.6,0}
\definecolor{codegray}{rgb}{0.5,0.5,0.5}
\definecolor{codepurple}{rgb}{0.58,0,0.82}
\definecolor{codebackground}{rgb}{0.95,0.95,0.92}

% style of the code!
\lstdefinestyle{codestyle}{
    backgroundcolor=\color{codebackground},   
    commentstyle=\color{codegreen},
    keywordstyle=\color{magenta},
    numberstyle=\tiny\color{codegray},
    stringstyle=\color{codepurple},
    basicstyle=\ttfamily\footnotesize,
    frame=single,
    breakatwhitespace=false,         
    breaklines=true,                 
    captionpos=b,                    
    keepspaces=true,                 
    numbers=left,                    
    numbersep=5pt,                  
    showspaces=false,                
    showstringspaces=false,
    showtabs=false,                  
    tabsize=2,
    title=\lstname 
}

\lstset{style=codestyle}


%% ---------------------------------------------------------------------------
% First frame (with tile, subtitle, ...)
\begin{frame}{}
    \maketitle
\end{frame}

%% ---------------------------------------------------------------------------
% Second frame
\begin{frame}{Sumário}
    \tableofcontents
\end{frame}

%% ---------------------------------------------------------------------------
% This presentation is separated by sections and subsections
\section{Seção I}
\begin{frame}{Explicações}
    % itemize
    Este é um template que pode ser utilizado para:
    \begin{itemize}
        \item Apresentação de Trabalhos Acadêmicos
        \item Apresentação de Disciplinas
        \item Apresentações de Teses e Dissertações
    \end{itemize}

    \vspace{0.4cm} % vertical space
    
    % enumeration
    Para utilizar este template corretamente é importante que:
    \begin{enumerate}
        \item Ter conhecimento mínimo sobre LaTeX
        \item Ler os comentários do template (explicações)
        \item Ler o README.md (documentação)
    \end{enumerate}

    \vspace{0.2cm}

    \example{Este é um texto de exemplo!} \emph{Texto de Ênfase!}
\end{frame}

%% ---------------------------------------------------------------------------
\subsection{Subseção I}
\begin{frame}{Criando Blocos}
    % Blocks styles
    \begin{block}{Meu Bloco Padrão}
        Texto do corpo do bloco.
    \end{block}

    \begin{alertblock}{Bloco de Alerta}
        Texto do corpo do bloco.
    \end{alertblock}

    \begin{exampleblock}{Bloco de Exemplo}
        Texto do corpo do bloco.
    \end{exampleblock}   
\end{frame}

%% ---------------------------------------------------------------------------
\subsection{Subseção II}
\begin{frame}{Criando Caixas}
    \successbox{testando o success box}

    \pause

    \alertbox{testando o alert box}

    \pause

    \examplebox{testando o example box}
\end{frame}

%% ---------------------------------------------------------------------------
\subsection{Subseção III}
\begin{frame}{Criando Algoritmos (Pseudocódigo)}
    \begin{algorithm}[H]
        \SetAlgoLined
        \LinesNumbered
        \SetKwInOut{Input}{input}
        \SetKwInOut{Output}{output}
        \Input{x: float, y: float}
        \Output{r: float}
        \While{True}{
          r = x + y\;
          \eIf{r >= 30}{
           ``O valor de $r$ é maior ou iqual a 10.''\;
           break\;
           }{
           ``O valor de $r$ = '', r\;
          }
         } 
         \caption{Algorithm Example}
    \end{algorithm}
\end{frame}

%% ---------------------------------------------
\begin{frame}{Inserindo Algoritmos}
    \lstset{language=Python}
    \lstinputlisting[language=Python]{code/main.py}
\end{frame}

%% ---------------------------------------------
\begin{frame}{Inserindo Algoritmos}
    \lstinputlisting[language=C]{code/source.c}
\end{frame}

%% ---------------------------------------------------------------------------
\begin{frame}{Inserindo Algoritmos}
    \lstinputlisting[language=Java]{code/helloworld.java}
\end{frame}

%% ---------------------------------------------
\begin{frame}{Inserindo Algoritmos}
    \lstinputlisting[language=HTML]{code/index.html}
\end{frame}

\end{document}